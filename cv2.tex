
\documentclass[oneside,a4paper,12pt]{article}  % velkost strany A4, velkost pisma, typ formatovania pre clanok/article

\usepackage[utf8]{inputenc}  % prikaz inputenc zadefinuje balicek pre kodovanie utf8 kvoli makcenom a dlznom

\usepackage[slovak]{babel}  % definuje balicek na generovanie textu v slovencine

\usepackage{amsmath}  % matematicke funkcie

\usepackage{graphicx}  % praca s obrazkami



% vytvorenie titulnej strany

\title  {\ Nazov temy}  % nazov  
\date {28-09-2020}  % datum (bez specifikovania generuje aktualny datum)

\author{Meno Priezvisko\\[2pt]   % meno autora
                    {\small Slovenska technicka univerzita Bratislave}\\
                    {\small Fakulta informatiky a informacnych technologii}\\
                    {\small \texttt{...@stuba.sk}}
            }


% zaciatok dokumentu

\begin{document} 

          \pagenumbering{gobble}  % necisluj prvu stranu
          \maketitle  % vytvor titulnu stranu
          \newpage  %vloz novu stranu
          \pagenumbering{arabic}  % pouzivaj arabske cislice
                            
 
 % vytvorenie sekcie, cislovanie je automaticke                           
\section{Kapitola}

          \begin{align*}  % zaciatok bloku matematicych operacii
                    1 + 2 &= 3\\   % \\ - dalsia operacia bude na novej strane 
                    1 &= 3 - 2                
          \end{align*}  % koniec bloku matematickych operacii


% vytvorenie podsekcie
\subsection{Podkapitola}

          \begin{align*}
                    f(x) &= x^2\\
                    g(x) &= \frac{1}{x}\\
                    F(x) &= \int^a_b \frac{1}{3}x^3
          \end{align*}


% vytvorenie podpodsekcie
\subsubsection{Podpodkapitola}

          \begin{figure}[htbp]  % oznacuje pracu s obrazkami, h/t/b/p hovoria o polohe obrazka v ramci textu

                    \centering  % zarovna obrazok na stred
                    \includegraphics[scale=0.20]{Nazov obrazka}  % scale hovori kolkokrat ma byt obrazok vacsi/mensi oproti povodnemu, meno obrazka - obr ma byt ulozeny v rovnakom priecinku
                    \caption{Nas prvy obrazok}  % oznacenie obrazka - popis 

          \end{figure}  % koniec bloku s obrazkami



Text
  

\begin{table}[h!]  % praca s tabulkami, h zarovnava tabulku za vetu, za ktorou sa nachadza dany prikaz

          \begin{center}

                    \label{tab:Tabulka 1}  % oznacenie tabulky
    
                    \begin{tabular}{l|c|r}  % vykreslenie stlpcov, prvy vlavo, druhy uprostred, treti vpravo
                              \textbf{Hodnota 1} & \textbf{Hodnota 2} & \textbf{Hodnota 3}\\   % zadanie textu, dalsi text bude na novom riadku, vypis cez \text, bf oznacuje hrubku pisma
                              $\alpha$ & $\beta$ & $\gamma$ \\   % dalsi text 
                              
                              \hline  % vykresli ciaru oddelujucu hlavicu od ostatneho textu
                              1 & 1110.1 & a\\  % vypln tabulky
                              2 & 10.1 & b\\
                              3 & 23.113231 & c\\
               
                    \end{tabular}  % koniec bloku s tabulkami
    
          \end{center} % ukoncnie zarovnania na stred
                    
          \caption{Nasa prva tabulka}  % pomenovanie tabulky
     
\end{table}  % koniec bloku s tabulkou

\end{document}  % koniec dokumentu